% !Mode:: "TeX:UTF-8"

\BiChapter{密集热点区域无线网络的优化}{UDN optimization}

根据第三章的分析,在不采用任何干扰管理和协调算法的情况下,网络中能达到正常工作的信干噪比的用户占总共的用户量的百分比很低。
从而导致了反应网络中所有用户有效性的平均的性能的单位面积谱效率也很低。
因此需要设计一种干扰管理和协调算法提升网络的性能。

本章从两个角度进行分析,即将分两步进行干扰管理和协调,第一步对小区中的微基站进行分簇,从而达到边缘用户减少,
第二步对每个簇中的基站采用~CRAN~网络架构进行联合,由于~CRAN~架构可以将不同微基站的基带信号传输至云化的~BBU~池进行统筹处理。
在~BBU~池侧,可以采用联合传输预编码的方法将发送给不同用户的信息在空域上相互正交,从而达到干扰消除的目的。

\BiSection{密集热点区域无线网络的干扰管理算法的架构}{algorithm construction}
根据图~\ref{e_capacity_show}~显示的结果,在小区中心的用户,收到的干扰较小,遍历容量较好,能达到正常通信的遍历容量需求。
小区边缘的用户的遍历容量较差,反应了小区边缘收到的干扰较为强烈,
接收信干噪比较低,处于边缘的终端出现中断的概率很大,因此要想办法去让边缘用户的性能提升,从而降低网络的中断概率,提高网络覆盖率。

联合传输技术是伴随第四代移动通信~(4G)~提出的一个关键技术。其主要的思想是以用户为中心,将基站联合起来,进行协作,共同对用户进行服务。
由于基站间相互联合共同服务区域内的用户用户,
处在基站边缘的用户的干扰功率也能被用户利用成为有用的接收功率,从而使得边缘用户的有效性大大提升。
联合传输的示意图如图~\ref{CoMP}~所示,在图中,网络由用于处理数字信号的BBU池,用户传输信号的微基站和前向回程链路组成。
传送给用户的信息首先通过BBU池进行数据处理,在通过前向回程链路传输到基站端,由于采用了干扰消除算法,因此两个基站传送给用户的信息均为用户可以利用的有效信号,
实现两个基站联合传输共同服务区域内的用户的目的。
\begin{figure}[htbp]
\centering
\includegraphics[width = 0.62\textwidth]{CoMP.pdf}
\caption{联合传输网络架构示意图}\vspace{-0.5em}
\label{CoMP}
\end{figure}

虽然联合传输可以技术可以提升边缘用户的通信性能,
但是联合传输技术并不能直接用于密集热点区域无线网络当中,密集热点区域无线网络,区域内的基站较为密集,基站的分布是不均匀的,
且一般情况下数量较多,如果将区域内的所有基站全部都联合在一起,系统的复杂度很高,实现难度太大。


为改进联合传输技术,使其不但能增加使处在服务区域边缘的用户的传输可靠性,提升网络的性能,又能有较低的复杂度。
本文提出的干扰管理算法分为两个步骤,首先,对小区中的微基站进行分簇,将簇内的基站整体作为协作集,共享信道的发送信息与信道状态信息。
然后再对每个簇中的基站采用~CRAN~网络架构进行联合,对同一个簇中的用户采用联合传输策略,
将簇内发送给各个用户的有用信号在空域上尽可能的正交,
增强发送给用户的有用信号,
抑制用户接收到的干扰,
达到增加网络的覆盖率,提升网络的区域面积频谱效率的目的。
网络的示意图如图~\ref{CoMP_cluster}~所示:
\begin{figure}[htbp]
\centering
\includegraphics[width = 0.7\textwidth]{CoMP_cluster.pdf}
\caption{密集热点区域无线网络的干扰管理架构示意图}\vspace{-0.5em}
\label{CoMP_cluster}
\end{figure}
在图中,将小区内的~5~个基站分成了三个簇,分别为簇~A,簇~B和簇~C,当簇内含有多个基站时采用~CRAN~网络架构,通过云服务器对发送的信号进行集中的数字信号处理。
对簇内采用干扰管理和协调算法,但不同簇的基站之间会产生簇间干扰。



\BiSection{密集热点区域无线网络的基站分簇}{clustering of basestation in UDN}

本小节对密集热点区域无线网络的基站的分簇算法进行研究,提出基于深度优先搜索的网络中微基站分簇算法。

\BiSubsection{基于深度优先搜索的微基站分簇算法}{Time diversity}
根据图~\ref{e_capacity_show}~所示,相邻的基站之间如果距离过近,则由于基站之间的干扰强烈,小区中用户的遍历容量性能,
网络的覆盖率和单位面积频谱效率性能将会受到巨大的影响,除此之外,如果两个基站的距离过近,由于微基站服务于用户量需求大,
容量要求高的区域,因此两个基站之间会存在大量的边缘用户,从而极大的影响了系统的性能。

为了避免基站相聚过近而导致的基站之间的干扰强烈,边缘用户过多。
本小节提出了基于深度优先搜索的基站分簇算法。

深度优先搜索算法是计算机科学当中的基础算法,也是图论当中比较经典的算法之一。

其主要的思想是在一个包含许多子图的图结构当中,已知图的邻接矩阵,通过遍历搜索的办法,
找到包含所有子图的集合。

为了应用按深度优先的搜索,首先需要在将网络中的基站拓扑映射成为一个图结构。

其中基站可以映射为图中的节点。
为确定两个基站的节点之间是否存在边。设定距离门限~$\tau$,对于任意的两个微基站~$S_i$,$S_j$,当两个基站之间的距离~$R_{i,j} > \tau$,则~$S_i$~和~$S_j$~之间存在~$e_{i,j}$。
根据基站节点的集合和表示节点之间是否为强干扰基站的节点之间的边,建立表示图邻接矩阵。
根据图的邻接矩阵,找到邻接矩阵所表示的图中的所有子图的集合,处于相同子图中的基站选择为协作的基站,从而完成对基站的分簇。

根据前面的概述,即可得到算法~\ref{algorithm_bs_dfs}。
\begin{table}[H]
\caption{基于深度优先搜索的基站分簇算法流程}
\label{algorithm_bs_dfs}
\vspace{0.5em}\wuhao
\begin{tabularx}{1.0\textwidth}{l}
\toprule[1.5pt]
算法~4-1~~~给予深度优先搜索的基站分簇算法 \\
\midrule[0.5pt]
1. 初始化:输入基站的位置信息,设定距离门限~$\tau$。 \\
2. 构造邻接矩阵:根据输入的基站的位置信息,距离的门限值~$\tau$构造邻接矩阵~$A$。\\
\qquad (1)~如果两个基站~$S_i$~和~$S_j$~之间的距离小于距离门限~$\tau$,则将邻接矩阵的相应位置置1:\\
\multicolumn{1}{c}{$a_{i,j}=a_{j,i}=1$}\\

 \qquad(2)~如果两个基站~$S_i$~和~$S_j$~之间的距离小于距离门限~$\tau$,则将邻接矩阵的相应位置置0:\\
\multicolumn{1}{c}{$a_{i,j}=a_{j,i}=0$}\\

3. 转换为非系统码:将系统码~${\bf{\tilde v}} = \left( {{\bf{\tilde b}},{\bf{\tilde u}}} \right)$~转换为满足~RC~约束条件的形式:\\
\multicolumn{1}{c}{${\bf{\tilde v}} = \left( {{{\tilde v}_0},{{\tilde v}_1}, \cdots ,{{\tilde v}_{2k - 1}}} \right)$} \\
4. SPA~译码:将修正后的信息位~LLR~${\hat L_{{{\tilde u}_i}}}$~和校验位~LLR~${\hat L_{{{\tilde b}_i}}}$~代入到~SPA~算法中,对~LDPC\\码进行译码。\\
\bottomrule[1.5pt]
\end{tabularx}
\end{table}

算法通过给定距离门限,将小于距离门限的所有基站连接在了一起。这些基站共同给所覆盖区域内的用户进行服务。
从而极大的降低了边缘用户的数量,增大了用户所接收到的有用功率,减小了用户接收到的干扰。

但是,算法本身还存在缺点,由于小区中基站的部署是泊松点过程,因此,小区中的基站的分布并不均匀,
因此可能会出现某个协作集当中的基站过多,
从而可能导致多用户簇内的多用户联合传输的复杂度过高。
不仅如此,由于小区中的微基站采用了以基站为中心的分簇方式。
小区中的服务范围依然存在明显的边界,处在边界上的用户依然会可能会受到比较严重的干扰。

\BiSubsection{以用户为中心的微基站分簇算法}{Time diversity}

\BiSubsection{空间分集方式}{Space diversity}



\BiSection{本章小结}{Conclusion}
